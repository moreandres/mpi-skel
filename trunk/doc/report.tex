\documentclass[a4paper]{report}

\usepackage{graphicx}
\usepackage{float}
\usepackage{hyperref}
\usepackage[spanish]{babel}

\begin{document}

\title{
  Universidad Nacional de La Plata\\Facultad de Inform\'atica\\
  \bigskip
  Especializaci\'on en C\'omputo de Altas Prestaciones\\
  \bigskip
  Herramientas para el Soporte de An\'alisis de Rendimiento
}

\author{
  Alumno: Andr\'es More - {\tt amore@hal.famaf.unc.edu.ar}\\
  Director: Dr Fernando G. Tinetti - {\tt fernando@lidi.info.unlp.edu.ar}
}

%\date{}

\maketitle

\begin{abstract}

  \bigskip

  Este documento describe una investigaci\'on realizada como trabajo final para
  la Especializaci\'on en C\'omputo de Altas Prestaciones dictada en la Facultad
  de Inform\'atica de la Universidad Nacional de La Plata.

  \bigskip

  El tema de investigaci\'on consiste en m\'etodos y herramientas para el soporte
  de an\'alisis de comportamiento en aplicaciones de alto rendimiento.

  \bigskip

  Luego de la introducci\'on de terminolog\'ia y bases te\'oricas del an\'alisis cuantitativo
  de rendimiento, se discute un modelo en particular de aplicaci\'on en etapas.

  \bigskip

  Se resume la experiencia de utilizar herramientas como {\it gprof} y {\it oprofile}
  para conocer donde se deber\'ia localizar los esfuerzos de optimizaci\'on.

\end{abstract}

\tableofcontents

\chapter{Introducci\'on}

Este cap\'itulo introduce este trabajo. Luego de revisar la motivaci\'on del mismo,
se resume el estado actual de la investigaci\'on y se detalla el contenido restante
del informe.

\section{Motivaci\'on}

En el \'area de c\'omputo de altas prestaciones los desarrolladores son en su mayor\'ia los especialistas del dominio del problema a investigar.
Este hecho resulta en menos tiempo para an\'alisis de resultados e impacta notablemente en el n\'umero de publicaciones de un grupo de investigaci\'on.

\bigskip

Las rutinas m\'as demandantes de c\'alculo son en su mayor\'ia cient\'ificas, su complejidad s\'olo hace posible su correcta implementaci\'on solo por especialistas en el dominio.
Con mayor impacto que en otras areas de la computaci\'on, el c\'odigo optimizado puede ejecutarse ordenes de magnitud mejor que implementaciones directas.

\bigskip

El proceso de optimizaci\'on de una implementaci\'on es entonces hecho de modo artesanal, sin mucha informaci\'on cuantitativa para dirigir los esfuerzos de optimizaci\'on.
Incluso se llega a preferir la implementaci\'on {\em ad-hoc} de algoritmos en lugar de la utilizaci\'on de librer\'ias ya disponibles optimizadas y con correctitud comprobada.

\section{Estado Actual}

Herramientas propietarias o muy espec\'ificas para una aplicaci\'on o sistema en particular. No aplicables en todos los casos, requiriendo instrumentaci\'on de c\'odigo.

\bigskip

Existen muchos patrones de dise\~no para implementaciones concurrentes o de altas prestaciones pero no son muy utilizados en este tipo de aplicaciones.

\section{Organizaci\'on del Contenido}

El resto del documento explica teor\'ia de an\'alisis de rendimiento y temas
relacionados. El capitulo 2 discute el analisis de rendimiento, sus principios y teorias.
El capitulo 3 detalla las herramientas mas utilizadas. El capitulo 4 ejemplifica la aplicacion de una herramienta para obtener informacion de analisis.
El capitulo 5 concluye y el capitulo 6 detalla posibles trabajos futuros.

\chapter{An\'alisis de Rendimiento}

Este cap\'itulo introduce el concepto de rendimiento y teor\'ia sobre su an\'alisis.

\section{Definici\'on}

El rendimiento se caracteriza por la cantidad de trabajo de c\'omputo que se logra
en comparaci\'on con la cantidad de tiempo y el uso de los recursos.

\section{M\'etricas}

Algunos ejemplos de medida de rendimiento son:

\begin{enumerate}
\item el ancho de banda y la latencia m\'inima de un canal de comunicaci\'on, una jerarqu\'ia de memorias o unidades de almacenamiento
\item la cantidad de instrucciones, operaciones, datos o trabajo a procesar por cierta unidad de tiempo
\item rendimiento por unidad de energia o por costo asociado
\end{enumerate}

Para medir el rendmiento se utilizan pruebas de referencias denominados {\em benchmarks}, que pueden ser aplicaciones sint\'eticas construidas especialmente para ejercitar ciertos
reecursos computacionales e incluso aplicaciones del mundo real bien conocidas.

\bigskip

Las caracter\'isticas deseables son portabilidad, simplicidad, estabilidad y reproducci\'on de resultados.
Esto permite que sean utilizadas para realizar mediciones cuantitativas y asi realizar comparaciones de optimizaciones o entre sistemas de c\'omputo diferentes.

\bigskip

Tambi\'en se pide que el tiempo de ejecuci\'on sea razonable y que el tama\~no del problema sea ajustable para poder seguir utilizandose con el paso del tiempo y el avance de las tecnologias.

\bigskip

A continuacion se introducen algunas de las m\'as utilizadas, junto con detalles y ejemplos de sus reportes.

\subsubsection{STREAM}

STREAM \cite{stream} es un benchmark sintetico a traves de un simple programa que mide el ancho de banda de memoria sostenido en MB/s y el rendimiento de computacion relativa de algunos vectores simples de calculo. Se utiliza para dimensionar el ancho de banda de acceso de escritura o lectura a la memoria principal del sistema bajo analisis.

\begin{verbatim}
-------------------------------------------------------------
STREAM version $Revision: 1.2 $
-------------------------------------------------------------
This system uses 8 bytes per DOUBLE PRECISION word.
-------------------------------------------------------------
Array size = 10000000, Offset = 0
Total memory required = 228.9 MB.
Each test is run 10 times, but only
the *best* time for each is used.
-------------------------------------------------------------
Number of Threads requested = 2
-------------------------------------------------------------
Printing one line per active thread....
-------------------------------------------------------------
Your clock granularity/precision appears to be 1 microseconds.
Each test below will take on the order of 39952 microseconds.
   (= 39952 clock ticks)
Increase the size of the arrays if this shows that
you are not getting at least 20 clock ticks per test.
-------------------------------------------------------------
Function      Rate (MB/s)   Avg time     Min time     Max time
Copy:        4764.1905       0.0337       0.0336       0.0340
Scale:       4760.2029       0.0338       0.0336       0.0340
Add:         4993.8631       0.0488       0.0481       0.0503
Triad:       5051.5778       0.0488       0.0475       0.0500
-------------------------------------------------------------
Solution Validates
-------------------------------------------------------------
\end{verbatim}

\subsubsection{HPL}

multiplicacion de matrices.

\subsection{IMB}

ping pong.

maximo ancho de banda. mensajes con datos grandes.
minima latencia. mensajes sin datos.

\subsubsection{HPCC}

conjunto de m\'ultiple micro benchmarks.
STREAM, HPL, ancho de banda y latencia, transformadas de {\it fourier}.

\subsubsection{DEISA}

un conjunto de aplicaciones de astrofisica, dinamica de fluidos, modelado climatico,
bio-ciencia, ciencia de los materiales, fusion de energia y fisica de particulas.

\bigskip

$ t = N * C / f $

\section{T\'ecnicas}

analisis de calidad de acuerdo a la ingenier\'ia del software.

\bigskip

el procedimiento usualmente consiste en medir, localizar, optimizar, comparar.
disciplina en un cambio a la vez asegura resultados reproducibles.

\bigskip

la reproduccion de resultados es compleja, en el caso de no tener una configuracion
de sistema estable en el tiempo, es recomendable siempre ejecutar una version
optimizada contra una version de referencia en un sistema de computo.

\bigskip

cuellos de botella. overhead. problemas de balanceo, contencion o uso de recursos.

\section{Modelos}

paralelizacion. niveles. bit/instruccion/datos/tareas.

calculo de mejora. teorica versus real. leyes de amdalah \cite{amdahl}, gustafson
\cite{gustafson} y karp-flatt \cite{karp-flatt}.

pipeline. formulas. rangos de optimizacion posibles.

\chapter{Herramientas}

Este cap\'itulo revisa.

\section{Taxonom\'ia}

din\'amico versus est\'atico.
tiempo de compilaci\'on versus tiempo de ejecucion.

\section{gprof}

profiling din\'amico.

se necesita compilar la aplicaci\'on con una opci\'on espec\'ifica.

se necesita ejecutar la aplicaci\'on con un conjunto de datos dado.

datos del perfil de una aplicaci\'on.

se necesita ejecutar un analizador sobre los datos acumulados.

perfil plano. una lista de las funciones ejecutadas ordenadas por la cantidad
acumulada de tiempo utilizado.

el gr\'afico de llamadas. muestra el tiempo utilizado por las funciones y sus hijos.

las funciones recursivas son manejadas de manera especial.

\subsection{perfil de ejecuci\'on}

time, cumulative seconds, self seconds, calls, self ms/call, total ms/call, name

\subsection{gr\'afico de llamadas}

index, \% time, self, children, called, name

\subsection{comportamiento}

precisi\'on estad\'istica. muestreo. incompatibilidades.

\subsection{ejemplos}

\section{oprofile}

\subsection{introduccion}

historia. resumen. caracter\'isticas. 

\subsection{procedimiento}

ejecutar el profiler. ejecutar la aplicaci\'on. generar el resumen.

no se necesita el c\'odigo.

c\'odigo anotado si hay s\'imbolos.

componentes del sistema.

contadores de performance.

costo adicional. overhead.

\subsection{contadores de rendimiento}

\subsection{ejemplos}

\chapter{Casos de Estudio}

Este cap\'itulo aplica.

\section{Multiplicaci\'on de Matrices}

La multiplicaci\'on de matrices es una operaci\'on fundamental en m\'ultiples campos
de aplicaci\'on cient\'ifica como la resoluci\'on de ecuaciones lineales y la
representaci\'on de grafos y espacios dimensionales. Por ello existe abundante
 material sobre el tema.

\bigskip

El c\'odigo fuente de una implementaci\'on simplista se encuentra adjuntado en el
ap\'endice. Al aplicar las herramientas vistas previamente se identifica claramente
que la multiplicaci\'on de los elementos de la matriz consume el mayor tiempo de
c\'omputo.

\bigskip

A continuaci\'on se muestra una comparaci\'on de diferentes m\'etodos, se demuestra
claramente con este ejercicio la sofisticaci\'on de librer\'ias contra m\'etodos
artesanales de optimizaci\'on.

\section{Distribuci\'on de Calor en Dos Dimensiones}

\section{Reinas}

\section{Transformadas de {\it Fourier}}

El c\'odigo fuente de una implementaci\'on simplista se encuentra adjuntado en el
ap\'endice. Al aplicar las herramientas vistas previamente se identifica claramente
que la evaluacion de derivadas parciales consume el mayor tiempo de c\'omputo.

\bigskip

A continuaci\'on se muestra una comparaci\'on de diferentes m\'etodos, se demuestra
claramente con este ejercicio la sofisticaci\'on de librer\'ias contra m\'etodos
artesanales de optimizaci\'on.

\chapter{Conclusiones}

Este cap\'itulo concluye.

Aportes: un m\'etodo, el an\'alisis de diferentes herramientas. resumen
introductorio del tema. ejercicios que demuestran que la optimizaci\'on artesanal
no es buena.

\chapter{Trabajo Futuro}

Este cap\'itulo propone.

una librer\'ia m\'as infrastructura para soporte de an\'alisis.
generaci\'on de reporte de rendimiento autom\'atico para guiar optimizaciones.
an\'alisis de mejora posible seg\'un formulas aplicadas a etapas.

\begin{thebibliography}{9}

\bibitem{beowulf}
 T. Sterling, D. Savarese, D. J. Becker, J. E. Dorband, U. A. Ranawake, and C. V. Packer,
 \emph{Beowulf: A parallel workstation for scientific computation},
 1995.

\bibitem{mpi}
  Message Passing Interface Forum,
  \emph{MPI: A Message-Passing Interface Standard},
  2.2,
  2009.

\bibitem{openmp}
  OpenMP Architecture Review Board,
  \emph{OpenMP Application Program Interface}.
  3.0,
  2008.

\bibitem{tinetti}
  Fernando G Tinetti,
  \emph{C\'omputo Paralelo en Redes Locales de Computadoras},
  2004.

\bibitem{gprof}
 Susan L. Graham,  Peter B. Kessler,  Marshall K. McKusick,
 \emph{gprof: A Call Graph Execution Profiler},
 1982.

\bibitem{oprofile}
J. Levon,
\emph{oprofile: hardware profiler for Linux systems},
{\tt http://oprofile.sourceforge.net}.

\bibitem{hennessy-patterson}
 John. L. Hennesy, David A. Patterson,
 \emph{Computer Architecture: A Quantitative Approach, 3rd Edition},
 2002.

\bibitem{intel}
 Intel Press,
 \emph{Intel64 and IA-32 Architectures Software Developer's Manual - Volume 3B: System Programming Guide, Part 2},
 March 2010.

\bibitem{what}
 Ulrich Deeper,
 \emph{What Every Programmer Should Know About Memory},
 November 2007.

\bibitem{patterns}
 G. Mattson, B.A. Sanders and B.L. Massingill, 
 \emph{Patterns for Parallel Programming, Addison-Wesley},
 2004.

\bibitem{automatic-performance-analysis}
 T. Margalef, J. Jorba, O. Morajko, A. Morajko, E. Luque,
 \emph{Different approaches to automatic performance analysis of distributed applications},
 2004.

\bibitem{capturing-performance-knowledge}
 K. Huck, O. Hernandez, V. Bui, S. Chandrasekaran, B. Chapman, A. Malony, L McInnes, B. Norris,
 \emph{Capturing performance knowledge for automated analysis},
 2008.

\bibitem{automatic-openmp-mpi-analysis}
 F. Wolf, B. Mohr,
 \emph{Automatic performance analysis of hybrid MPI/OpenMP applications},
 2003.

\bibitem{intro-software-performance}
 C. Smith,
 \emph{Introduction to software performance engineering: origins and outstanding problems},
 2007.

\bibitem{future-software-performance}
 M. Woodside, G. Franks, D. Petriu,
 \emph{The Future of Software Performance Engineering},
 2007.

\bibitem{critical-overview}
 J. Browne,
 \emph{A critical overview of computer performance evaluation},
 1976.

\bibitem{hpctoolkit}
  Rice University,
 \emph{HPC Toolkit},
 {\tt http://hpctoolkit.org}.

\bibitem{papi}
  University of Tennessee,
  \emph{Performance Application Programming Interface},
  {\tt http://icl.cs.utk.edu/papi}.

\bibitem{amdahl}
  G. M. Amdahl,
  \emph{Validity of single-processor approach to achieving large-scale computing
    capability},
  Proceedings of AFIPS Conference, Reston, VA. 1967. pp. 483-485.

\bibitem{twelve-ways}
  D. Bailey, \emph{Twelve Ways to Fool the Masses When Giving Performance Results
    on Parallel Computers},
  RNR Technical Report, RNR-90-020, NASA Ames Research Center, 1991.

\bibitem{gustafson}
  J. L. Gustafson,
  \emph{Reevaluating Amdahl's Law}, CACM, 31(5), 1988. pp. 532-533.

\bibitem{karp-flatt}
  A. H. Karp and H. P. Flatt,
  \emph{Measuring Parallel Processor Performance},
  Communication of the ACM Volume 33 Number 5, May 1990.

\bibitem{myth}
 {Lee, Victor W. and Kim, Changkyu and Chhugani, Jatin and Deisher, Michael and Kim, Daehyun and Nguyen, Anthony D. and Satish, Nadathur and Smelyanskiy, Mikhail and Chennupaty}, Srinivas, and Hammarlund, Per and Singhal, Ronak and Dubey, Pradeep,
 \emph{Debunking the 100X GPU vs. CPU myth: an evaluation of throughput computing on CPU and GPU},
 2010.

\bibitem{stream}
 John D. McCalpin,
 \emph{A Survey of Memory Bandwidth and Machine Balance in Current High Performance Computers},
 1995.

\end{thebibliography}

\end{document}

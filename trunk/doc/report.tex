\documentclass[a4paper]{report}

\usepackage{graphicx}
\usepackage{float}
\usepackage{hyperref}
\usepackage[spanish]{babel}

\begin{document}

\title{Universidad Nacional de La Plata - Facultad de Inform\'atica\\
Especializaci\'on en C\'omputo de Altas Prestaciones\\
\bigskip
Herramientas para el Soporte de An\'alisis de Rendimiento}

\author{Alumno: Andr\'es More - {\tt amore@hal.famaf.unc.edu.ar}\\
Director: Fernando Tinetti - {\tt fernando@lidi.info.unlp.edu.ar}}

\date{Abril 2011}
\maketitle

\begin{abstract}
  Este documento describe una investigaci\'on realizada como trabajo final para
  la Especializaci\'on en C\'omputo de Altas Prestaciones dictada en la Facultad
  de Inform\'atica de la Universidad Nacional de La Plata.
\end{abstract}

\tableofcontents

\chapter{Introducci\'on}

Este cap\'itulo introduce.

\section{Motivaci\'on}

Los desarrolladores son especialistas del dominio.

Menos tiempo para resultados y publicaciones.

Optimizaci\'on artesanal.

\section{Estado Actual}

Herramientas propietarias.

\section{Organizaci\'on del Contenido}

\chapter{Rendimiento}

Este cap\'itulo resume.

\section{Definici\'on}

\chapter{An\'alisis de Rendimiento}

Este cap\'itulo describe.

\section{Modelos}

\chapter{T\'ecnicas}

Este cap\'itulo revisa.

\section{Taxonom\'ia}

\chapter{Herramientas}

Este cap\'itulo revisa.

\section{Taxonom\'ia}

din\'amico versus est\'atico.

tiempo de compilaci\'on.

\section{gprof}

profiling din\'amico.

se necesita compilar la aplicaci\'on con una opci\'on espec\'ifica.

se necesita ejecutar la aplicaci\'on con un conjunto de datos dado.

datos del perfil de una aplicaci\'on.

se necesita ejecutar un analizador sobre los datos acumulados.

perfil plano. una lista de las funciones ejecutadas ordenadas por la cantidad
acumulada de tiempo utilizado.

el gr\'afico de llamadas. muestra el tiempo utilizado por las funciones y sus hijos.

las funciones recursivas son manejadas de manera especial.

\subsection{perfil de ejecuci\'on}

time, cumulative seconds, self seconds, calls, self ms/call, total ms/call, name

\subsection{gr\'afico de llamadas}

index, \% time, self, children, called, name

\subsection{comportamiento}

precisi\'on estad\'istica. muestreo. incompatibilidades.

\subsection{ejemplos}

\section{oprofile}

\subsection{introduccion}

historia. resumen. caracter\'isticas. 

\subsection{procedimiento}

ejecutar el profiler. ejecutar la aplicaci\'on. generar el resumen.

no se necesita el c\'odigo.

c\'odigo anotado si hay s\'imbolos.

componentes del sistema.

contadores de performance.

costo adicional. overhead.

\subsection{contadores de rendimiento}

\subsection{ejemplos}

\chapter{Casos de Estudio}

Este cap\'itulo aplica.

\section{Multiplicaci\'on de Matrices}

\section{Distribuci\'on de Calor 2D}

\section{Reinas}

\section{Transformadas de {\it Fourier}}

\chapter{Conclusiones}

Este capitulo concluye.

\chapter{Trabajo Futuro}

Este capitulo propone.

\begin{thebibliography}{9}
  
\bibitem{mpi}
  Message Passing Interface Forum,
  \emph{MPI: A Message-Passing Interface Standard},
  2.2,
  2009.

\bibitem{openmp}
  OpenMP Architecture Review Board,
  \emph{OpenMP Application Program Interface}.
  3.0,
  2008.

\bibitem{tinetti}
  Fernando Tinetti,
  \emph{C\'omputo Paralelo en Redes Locales de Computadoras},
  2004.

\bibitem{gprof}
 Susan L. Graham,  Peter B. Kessler,  Marshall K. McKusick,
 \emph{gprof: A Call Graph Execution Profiler},
 1982.

\bibitem{oprofile}
J. Levon,
\emph{oprofile: hardware profiler for Linux systems},
{\tt http://oprofile.sourceforge.net}.

\bibitem{hennessy-patterson}
 John. L. Hennesy, David A. Patterson,
 \emph{Computer Architecture: A Quantitative Approach, 3rd Edition},
 2002.

\bibitem{intel}
 Intel Press,
 \emph{Intel64 and IA-32 Architectures Software Developer's Manual - Volume 3B: System Programming Guide, Part 2},
 March 2010.

\bibitem{what}
 Ulrich Deeper,
 \emph{What Every Programmer Should Know About Memory},
 November 2007.

\bibitem{patterns}
 G. Mattson, B.A. Sanders and B.L. Massingill, 
 \emph{Patterns for Parallel Programming, Addison-Wesley},
 2004.

\bibitem{automatic-performance-analysis}
 T. Margalef, J. Jorba, O. Morajko, A. Morajko, E. Luque,
 \emph{Different approaches to automatic performance analysis of distributed applications},
 2004.

\bibitem{capturing-performance-knowledge}
 K. Huck, O. Hernandez, V. Bui, S. Chandrasekaran, B. Chapman, A. Malony, L McInnes, B. Norris,
 \emph{Capturing performance knowledge for automated analysis},
 2008.

\bibitem{automatic-openmp-mpi-analysis}
 F. Wolf, B. Mohr,
 \emph{Automatic performance analysis of hybrid MPI/OpenMP applications},
 2003.

\bibitem{intro-software-performance}
 C. Smith,
 \emph{Introduction to software performance engineering: origins and outstanding problems},
 2007.

\bibitem{future-software-performance}
 M. Woodside, G. Franks, D. Petriu,
 \emph{The Future of Software Performance Engineering},
 2007.

\bibitem{critical-overview}
 J. Browne,
 \emph{A critical overview of computer performance evaluation},
 1976.

\bibitem{hpctoolkit}
  Rice University,
 \emph{HPC Toolkit},
 {\tt http://hpctoolkit.org}.

\bibitem{papi}
  University of Tennessee,
  \emph{Performance Application Programming Interface},
  {\tt http://icl.cs.utk.edu/papi}.

% http://en.wikipedia.org/wiki/Performance_tuning

\end{thebibliography}

\end{document}

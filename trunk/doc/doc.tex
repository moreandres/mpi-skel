\documentclass[a4paper,twocolumn]{article}

\usepackage{graphicx}
\usepackage{float}
\usepackage{hyperref}
\usepackage[spanish]{babel}

\begin{document}

\title{Universidad Nacional de La Plata - Facultad de Inform\'atica\\
Especializaci\'on en C\'omputo de Altas Prestaciones\\
\bigskip
Propuesta de Trabajo Final\\
Herramientas para el Soporte de An\'alisis de Rendimiento}

\author{Alumno: Andr\'es More - {\tt amore@hal.famaf.unc.edu.ar}\\
Director: Fernando Tinetti - {\tt fernando@lidi.info.unlp.edu.ar}}

\date{Agosto 2010}

\twocolumn[
  \begin{@twocolumnfalse}
    \maketitle
    \begin{abstract}

      Este documento describe una propuesta de Trabajo Final para la
      Especializaci\'on en C\'omputo de Altas Prestaciones dictada en la Facultad
      de Inform\'atica de la Universidad Nacional de La Plata.

      \smallskip

      La propuesta principal consiste en el estudio general de los m\'etodos existentes para
      evaluaci\'on y an\'alisis de aplicaciones de alto rendimiento.

      \smallskip

      Se establecen m\'ultiples objetivos espec\'ificos para
      este trabajo, incluyendo la documentaci\'on de requerimientos en este tipo de
      aplicaciones, la generaci\'on de una taxonom\'ia de las tecnolog\'ias
      disponibles, su posterior estudio y ejercitaci\'on, y finalmente el desarrollo de 
      un m\'etodo generalizado para an\'alisis de
      rendimiento.

      \smallskip

      Luego de cumplir estos objetivos se plantea como cierre analizar
      varios n\'ucleos de c\'alculo como un ejercicio
      integrador. De este modo se podr\'a demostrar el procedimiento
      de an\'alisis desarrollado utilizando las herramientas y metodolog\'ias
      estudiadas previamente.

      \bigskip

    \end{abstract}
  \end{@twocolumnfalse}
]

% \tableofcontents

\section{Motivaci\'on}

El desarrollo de aplicaciones que utilizan c\'omputo de alto rendimiento es
principalmente realizado por especialistas en el dominio cient\'ifico de los problemas
a simular o resolver, frecuentemente fuera del \'area de la inform\'atica.

\smallskip

Estos cient\'ificos no poseen todos los conocimientos, experiencia o recursos
necesarios para una implementaci\'on adecuada, implicando una reducci\'on del
tiempo disponible para el an\'alisis de los resultados obtenidos y su posterior publicaci\'on.

\smallskip

Al no existir un m\'etodo general para el an\'alisis de rendimiento los
desarrolladores suelen utilizar procedimientos artesanales, e incluso aplicar
optimizaciones sin tener las bases cuantitativas suficientes a la hora de dirigir sus esfuerzos.

\section{Estado Actual}

Actualmente existen diversas tecnolog\'ias para el an\'alisis de rendimiento,
ofreciendo diferentes niveles de portabilidad, intrusi\'on, aplicaci\'on o granularidad de la
informaci\'on obtenida.

\smallskip

La idea m\'as simple es utilizar el tiempo total de ejecuci\'on como medida de cuan
r\'apido se completa una tarea, aunque no se obtiene informaci\'on relevante sobre el lugar adecuado o
las consecuencias de potenciales optimizaciones.

\smallskip

Otra posibilidad es implementar en la aplicaci\'on misma un sistema interno de
administraci\'on de tiempo y recursos; en el caso de estar disponibles, tambi\'en se
pueden invocar a mecanismos de evaluaci\'on propuestos por librer\'ias de
terceros utilizadas dentro de la aplicaci\'on.

\smallskip

Una posibilidad m\'as interesante es utilizar un perfil de ejecuci\'on extra\'ido
din\'amicamente para entender que partes del programa son utilizadas, en que orden y cuales son
las que demandan mayor tiempo de ejecuci\'on.

\smallskip

En el otro extremo, el {\it hardware} que conforma una unidad de c\'omputo suele permitir
acceso a registros contadores de eventos internos que pueden ser accedidos para comprobar el
nivel de eficiencia de la ejecuci\'on de un programa.

\section{Objetivos}

Adem\'as del estudio general de la evaluaci\'on y an\'alisis de rendimiento, este trabajo tiene como
objetivos espec\'ificos las siguientes actividades:

\begin{enumerate}
\item Generar una taxonom\'ia de m\'etodos y herramientas de an\'alisis de rendimiento.
\item Ejercitar estas herramientas, documentando pasos, problemas comunes y comparando sus caracter\'isticas y limitaciones.
\item Desarrollar una propuesta de un proceso de an\'alisis para aplicaciones de altas prestaciones.
\item Desarrollar y analizar diferentes aplicaciones aplicando las tecnolog\'ias previamente estudiadas.
\end{enumerate}

\section{Temas a Investigar}

Esta secci\'on contiene una lista inicial de los temas a desarrollar.

\begin{enumerate}
\item Conceptos de an\'alisis de rendimiento: definiciones, m\'etricas, procesos.
\item Modelo de ejecuci\'on de tipo {\it pipeline}: teor\'ia y aplicaciones.
\item Herramientas: caracter\'isticas, limitaciones y comparativas.
\begin{enumerate}
\item {\it oprofile}: evaluaci\'on a nivel de sistema.
\item {\it gprof}: evaluaci\'on a nivel de aplicaci\'on.
\end{enumerate}
\item Comunicaci\'on:  evaluaci\'on de utilizaci\'on de ancho de banda y latencia de canales
\begin{enumerate}
\item Librer\'ias de Paso de Mensajes: interfaz de evaluacion de rendimiento, trazas de ejecuci\'on.
\end{enumerate}
\end{enumerate}

\section{Plan de Actividades}

El siguiente cronograma muestra el plan de actividades tentativo.

\begin{figure}[H]
  \begin{center}
    \begin{tabular}{|l|l|}\hline
      {\bf Actividad} & {\bf Duraci\'on} \\ \hline
      Recolecci\'on de Material & Agosto \\ \hline
      Tecnolog\'ias & Setiembre/Octubre \\ \hline
      Procedimiento General & Noviembre \\ \hline
      An\'alisis de Ejemplos & Diciembre 2010 \\ & Enero 2011 \\ \hline
      Resultados/Conclusiones & Febrero \\ \hline
    \end{tabular}
    \caption{Detalle de Actividades}
  \end{center}
  \label{schedule}
\end{figure}

\section{Trabajo Futuro}

Esta propuesta de trabajo es presentada como base a un trabajo posterior para alcanzar el grado de magister,
abarcando la construcci\'on conjunta de una librer\'ia m\'as una infrastructura automatizada de an\'alisis de rendimiento.

\smallskip

Una vez entendidas las posibilidades tecnol\'ogicas y estudiado su aplicaci\'on en
m\'ultiples casos de ejemplo se puede entonces emprender una implementaci\'on de los
requerimientos de este tipo de aplicaciones y del proceso mismo de an\'alisis.

\section{Material de Referencia}

Existe abundante material sobre el tema, desde textos base a documentaci\'on de tecnolog\'ias
o herramientas en particular. Un conjunto inicial a modo de referencia es
inclu\'ida en la bibliograf\'ia listada al final.

\appendix

\section{Ejemplos Integradores}

Utilizando las t\'ecnicas estudiadas, se propone implementar y luego analizar las
siguientes aplicaciones como cierre del trabajo.

\begin{itemize}

\item \emph{matrix}: multiplicaci\'on de dos matrices densas, utilizando operaciones
  de punto flotante en forma secuencial.

\item \emph{heat2d}: distribuci\'on de calor en dos dimensiones, utilizando
  iteraciones de operaciones simples que simulan el paso del tiempo.

\item \emph{fft}: aplicaci\'on de transformadas de {\it Fourier} en espacios tridimensionales, utilizando
  operaciones complejas de punto flotante.

\item \emph{queens}: colocaci\'on de reinas en un tablero de
  ajedrez sin amenaza mutua, utilizando operaciones de {\it backtracking} sobre arreglos de enteros.

\end{itemize}

\begin{thebibliography}{9}
  
\bibitem{mpi}
  Message Passing Interface Forum,
  \emph{MPI: A Message-Passing Interface Standard},
  2.2,
  2009.

\bibitem{openmp}
  OpenMP Architecture Review Board,
  \emph{OpenMP Application Program Interface}.
  3.0,
  2008.

\bibitem{tinetti}
  Fernando Tinetti,
  \emph{C\'omputo Paralelo en Redes Locales de Computadoras},
  2004.

\bibitem{gprof}
 Susan L. Graham,  Peter B. Kessler,  Marshall K. McKusick,
 \emph{gprof: A Call Graph Execution Profiler},
 1982.

\bibitem{oprofile}
J. Levon,
\emph{oprofile: hardware profiler for Linux systems},
{\tt http://oprofile.sourceforge.net}.

\bibitem{hennessy-patterson}
 John. L. Hennesy, David A. Patterson,
 \emph{Computer Architecture: A Quantitative Approach, 3rd Edition},
 2002.

\bibitem{intel}
 Intel Press,
 \emph{Intel64 and IA-32 Architectures Software Developer's Manual - Volume 3B: System Programming Guide, Part 2},
 March 2010.

\bibitem{what}
 Ulrich Deeper,
 \emph{What Every Programmer Should Know About Memory},
 November 2007.

\bibitem{patterns}
 G. Mattson, B.A. Sanders and B.L. Massingill, 
 \emph{Patterns for Parallel Programming, Addison-Wesley},
 2004.

\bibitem{automatic-performance-analysis}
 T. Margalef, J. Jorba, O. Morajko, A. Morajko, E. Luque,
 \emph{Different approaches to automatic performance analysis of distributed applications},
 2004.

\bibitem{capturing-performance-knowledge}
 K. Huck, O. Hernandez, V. Bui, S. Chandrasekaran, B. Chapman, A. Malony, L McInnes, B. Norris,
 \emph{Capturing performance knowledge for automated analysis},
 2008.

\bibitem{automatic-openmp-mpi-analysis}
 F. Wolf, B. Mohr,
 \emph{Automatic performance analysis of hybrid MPI/OpenMP applications},
 2003.

\bibitem{intro-software-performance}
 C. Smith,
 \emph{Introduction to software performance engineering: origins and outstanding problems},
 2007.

\bibitem{future-software-performance}
 M. Woodside, G. Franks, D. Petriu,
 \emph{The Future of Software Performance Engineering},
 2007.

\bibitem{critical-overview}
 J. Browne,
 \emph{A critical overview of computer performance evaluation},
 1976.

\bibitem{hpctoolkit}
  Rice University,
 \emph{HPC Toolkit},
 {\tt http://hpctoolkit.org}.

\bibitem{papi}
  University of Tennessee,
  \emph{Performance Application Programming Interface},
  {\tt http://icl.cs.utk.edu/papi}.

% http://en.wikipedia.org/wiki/Performance_tuning

\end{thebibliography}

\end{document}

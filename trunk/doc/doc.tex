\documentclass[a4paper,twocolumn]{article}

\usepackage{graphicx}
\usepackage{float}
\usepackage{hyperref}
\usepackage[spanish]{babel}

\begin{document}

\title{Facultad de Informatica - Universidad Nacional de La Plata\\Especializaci\'on en C\'omputo de Altas Prestaciones\\Propuesta de Trabajo Final\\Herramientas para el Soporte de Analisis de Rendimiento}
\author{Alumno: Andr\'es More - {\tt amore@hal.famaf.unc.edu.ar}\\Director: Fernando Tinetti - {\tt fernando@lidi.info.unlp.edu.ar}}
\date{Mayo 2009}

\twocolumn[
\begin{@twocolumnfalse}
\maketitle
\begin{abstract}
Este documento describe una propuesta de trabajo final para Especializacion en Computo de Altas Prestaciones.
Idea. Beneficios. Ejemplo.
\end{abstract}
\end{@twocolumnfalse}
]

\tableofcontents

\section{Objetivos}

Generar una taxonomia de metodos de analisis de rendimiento.
Realizar una comparacion de los mismos.
Ejercitar estas herramientas, documentando pasos y problemas comunes.
Proponer un proceso de analisis para aplicaciones de alto rendimiento.
Generar infrastructura que soporte automaticamente el analisis de performance.
Generar una libreria que facilite el desarollo de aplicaciones de altas prestaciones.
Documentar requerimientos generales de aplicacions de altas prestaciones.
Generalizar las aplicaciones utilizando un pipeline.

\section{Motivacion y Estado del Arte}

MPI.
Expertos en Dominio.
Analisis de Performance es artesanal.

\section{Temas a Investigar}

Analisis de Rendimiento.
Herramientas.

\section{Material Bibliografico}

La gran mayoria de las referencias son a su vez material a utilizar durante la investigacion.

\begin{enumerate}
\item{Design Patterns}. Base de todo estudio de patrones de dise\~no
\item{Linux Device Drivers}. Detalles de implementacion de subsistemas utilizando patrones de reuso.
\item{G. Mattson, B.A. Sanders and B.L. Massingill, Patterns for Parallel Programming, Addison-Wesley, 2004}.
\end{enumerate}

\section{Plan de Actividades}

\appendix

\section{Ejemplo Integrador}

\subsection{matrix}

Matrices.

\subsection{queens}

\subsection{heat2}

\begin{thebibliography}{9}
 
\bibitem{mpi}
  Message Passing Interface Forum,
  \emph{MPI: A Message-Passing Interface Standard},
  2.2,
  2009.

\bibitem{openmp}
  OpenMP Architecture Review Board,
  \emph{OpenMP Application Program Interface}.
  3.0,
  2008.

\bibitem{tinetti}
  Fernando Tinetti,
  \emph{Computo Paralelo en Redes Locales de Computadoras},
  2004.

\end{thebibliography}

\end{document}

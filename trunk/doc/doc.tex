\documentclass[a4paper,twocolumn]{article}

\usepackage{graphicx}
\usepackage{float}
\usepackage{hyperref}
\usepackage[spanish]{babel}

\begin{document}

\title{Universidad Nacional de La Plata - Facultad de Inform\'atica\\
Especializaci\'on en C\'omputo de Altas Prestaciones\\
\bigskip
Propuesta de Trabajo Final\\
Herramientas para el Soporte de An\'alisis de Rendimiento}

\author{Alumno: Andr\'es More - {\tt amore@hal.famaf.unc.edu.ar}\\
Director: Fernando Tinetti - {\tt fernando@lidi.info.unlp.edu.ar}}

\date{Agosto 2010}

\twocolumn[
  \begin{@twocolumnfalse}
    \maketitle
    \begin{abstract}
      Este documento describe una propuesta de Trabajo Final para la
      Especializaci\'on en C\'omputo de Altas Prestaciones, dictada en la Facultad
      de Inform\'atica de la Universidad Nacional de La Plata. La propuesta
      consiste en el estudio de los m\'etodos existentes para evaluaci\'on y
      optimizaci\'on del rendimiento de programas. M\'ultiples objetivos
      inter-relacionados son establecidos para este trabajo, incluyendo la
      documentaci\'on de las requerimientos comunes en este tipo de aplicaciones,
      la generaci\'on de una taxonom\'ia de las tecnolog\'ias disponibles,
      su ejercitaci\'on, y una propuesta de un procedimiento generalizado para
      an\'alisis de rendimiento, entre otros.  Luego de cumplir los objetivos
      iniciales se plantea analizar varios n\'ucleos espec\'ificos de c\'alculo
      como casos integradores. De este modo se podr\'a demostrar el procedimiento
      de an\'alisis de rendimiento utilizando las herramientas y metodolog\'ias
      estudiadas.
    \end{abstract}
  \end{@twocolumnfalse}
]

\tableofcontents

\section{Motivaci\'on}

El desarrollo de aplicaciones que utilizan c\'omputo de alto rendimiento es
frecuentemente hecho por especialistas en el dominio de los problemas a resolver,
fuera del \'area de la inform\'atica.

Estos cient\'ificos no poseen todos los conocimientos, experiencia o recursos
necesarios para una implementaci\'on adecuada, implicando una reducci\'on del
tiempo disponible para el an\'alisis de los resultados de los c\'alculos o
simulaciones.

Al mismo tiempo, el desarrollo de estas aplicaciones suele ser para un uso
espec\'ifico y limitado, por lo que no existe un m\'etodo general para el
an\'alisis de su rendimiento. Forzando a los desarrolladores de tales aplicaciones
al uso de m\'etodos artesanales o a una optimizaci\'on sin tener las bases
cuantitativas suficientes.

\section{Estado del Arte}

Actualmente existen diversos m\'etodos y tecnolog\'ias para el an\'alisis de
rendimiento.

Lo m\'as simple es utilizar el tiempo total de ejecuci\'on. O medir el uso de
recursos del sistema durante la ejecuci\'on de la aplicaci\'on.

Existen multiples conceptos a considerar.
An\'alisis est\'atico o din\'amico.
Instrumentacion de c\'odigo.
Analisis desde Espacio de Usuario.
Rendimiento del Sistema Operativo.
Registros Contadores de Hardware. Registros de CPUs.

\section{Objetivos}

Generar una taxonom\'ia de m\'etodos de analisis de rendimiento.
Realizar una comparaci\'on de los mismos.
Ejercitar estas herramientas, documentando pasos y problemas comunes.
Proponer un proceso de analisis para aplicaciones de alto rendimiento.
Documentar requerimientos generales de aplicacioness de altas prestaciones.

\section{Temas a Investigar}

Analisis de Rendimiento.
Scaling
Pipeline.

Herramientas.

oprofile \cite{oprofile}

gprof \cite{gprof}

mpi profiling interface, traces. \cite{mpi}

Comunicaci\'on: Balance, Comunicaci\'on.

Mejora potencial seg\'un configuraci\'on de hardware.

\section{Plan de Actividades}

\begin{figure}[H]
  \begin{center}
    \begin{tabular}{|c|c|}\hline
      {\bf Actividad} & {\bf Duraci\'on} \\ \hline
      Recolecci\'on de Material & Julio \\ \hline
      Tecnolog\'ias & Agosto \\ \hline
      Procedimiento Generalizado & Septiembre \\ \hline
      Desarrollo/An\'alisis de Ejemplos & Octubre \\ \hline
      Resultados y Conclusiones & Noviembre \\ \hline
    \end{tabular}
    \caption{Detalle de Actividades}
  \end{center}
  \label{schedule}
\end{figure}

\section{Trabajo Futuro}

Esqueleto, patr\'on de dise\~no.

Pipeline Generalizaci\'on.

Automatizaci\'on.

Callbacks LDD.

Generalizar las aplicaciones utilizando un pipeline.

Generar infrastructura que soporte automaticamente el analisis de performance.

Generar una librer\'ia que facilite el desarollo de aplicaciones de altas prestaciones.

\section{Material de Referencia}

Citaciones de Libros, Articulos en las referencias.

Existe mucho material sobre el tema.
Articulos sobre aplicaciones especificas para mdir rendimiento.
Articulos sobre 

\appendix

\section{Ejemplos Integradores}

La aplicaci\'on \emph{matrix} realizara una multiplicacion de dos matrices densas.
operaciones de punto flotante en forma secuencial.

La aplicaci\'on \emph{queens} realizara la colocacion de 8 reinas en un tablero de
ajedrez sin que las mismas se amenacen mutuamente. arreglos de enteros.

\emph{heat2d} Distribuci\'on de calor en dos dimensiones.
operaciones simples en iteraciones.

\emph{fft} Utilizacion de transformadas de fourier.
operaciones altamente complejas de punto flotante.

\begin{thebibliography}{9}
  
\bibitem{mpi}
  Message Passing Interface Forum,
  \emph{MPI: A Message-Passing Interface Standard},
  2.2,
  2009.

\bibitem{openmp}
  OpenMP Architecture Review Board,
  \emph{OpenMP Application Program Interface}.
  3.0,
  2008.

\bibitem{tinetti}
  Fernando Tinetti,
  \emph{Computo Paralelo en Redes Locales de Computadoras},
  2004.

\bibitem{gprof}
 Susan L. Graham,  Peter B. Kessler,  Marshall K. McKusick,
 \emph{gprof: A Call Graph Execution Profiler},
 1982.

\bibitem{oprofile}
J. Levon,
\emph{oprofile: hardware profiler for Linux systems},
{\tt http://oprofile.sourceforge.net}.

\bibitem{hennessy-patterson}
 John. L. Hennesy, David A. Patterson,
 \emph{Computer Architecture: A Quantitative Approach, 3rd Edition},
 2002.

\bibitem{intel}
 Intel Press,
 \emph{Intel64 and IA-32 Architectures Software Developer's Manual - Volume 3B: System Programming Guide, Part 2},
 March 2010.

\bibitem{what}
 Ulrich Deeper,
 \emph{What Every Programmer Should Know About Memory},
 Novermber 2007.

\bibitem{patterns}
 G. Mattson, B.A. Sanders and B.L. Massingill, 
 \emph{Patterns for Parallel Programming, Addison-Wesley},
 2004.

\bibitem{automatic-performance-analysis}
 T. Margalef, J. Jorba, O. Morajko, A. Morajko, E. Luque,
 \emph{Different approaches to automatic performance analysis of distributed applications},
 2004.

\bibitem{capturing-performance-knowledge}
 K. Huck, O. Hernandez, V. Bui, S. Chandrasekaran, B. Chapman, A. Malony, L McInnes, B. Norris,
 \emph{Capturing performance knowledge for automated analysis},
 2008.

\bibitem{automatic-openmp-mpi-analysis}
 F. Wolf, B. Mohr,
 \emph{Automatic performance analysis of hybrid MPI/OpenMP applications},
 2003.

\bibitem{intro-software-performance}
 C. Smith,
 \emph{Introduction to software performance engineering: origins and outstanding problems},
 2007.

\bibitem{future-software-performance}
 M. Woodside, G. Franks, D. Petriu,
 \emph{The Future of Software Performance Engineering},
 2007.

\bibitem{critical-overview}
 J. Browne,
 \emph{A critical overview of computer performance evaluation},
 1976.

\bibitem{hpctoolkit}
  Rice University,
 \emph{HPC Toolkit},
 {\tt http://hpctoolkit.org}.

\bibitem{papi}
  TBD,
  \emph{Performance Application Programming Interface},
  {\tt http://icl.cs.utk.edu/papi}.

% http://en.wikipedia.org/wiki/Performance_tuning

\end{thebibliography}

\end{document}

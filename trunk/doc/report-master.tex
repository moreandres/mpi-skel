\documentclass[a4paper]{report}

\usepackage{graphicx}
\usepackage{float}
\usepackage{hyperref}
\usepackage{xcolor}
\usepackage[spanish]{babel}
\usepackage{listings}
\usepackage{enumitem}
\usepackage[utf8]{inputenc}

\lstset{language=c, frame=tlrb, basicstyle=\scriptsize, breaklines=true, numberbychapter=false,numbers=left}
\setlist[enumerate]{noitemsep}
\setlist[itemize]{noitemsep}

\bibliographystyle{unsrt}

\begin{document}

\renewcommand{\tablename}{Tabla}

\title{Universidad Nacional de La Plata\\Facultad de Informática\\ \bigskip
{\large Tesis presentada para obtener el grado de \\}  Magister en Cómputo de Altas Prestaciones\\ \bigskip
  Infraestructura para el Soporte de Análisis de Rendimiento}

\author{
  Alumno: Andrés More - {\tt amore@hal.famaf.unc.edu.ar}\\
  Director: Dr Fernando G. Tinetti - {\tt fernando@lidi.info.unlp.edu.ar}
}

\date{Octubre de 2014}

\maketitle

\begin{abstract}
one-liner marketinero.

contribuciones.

mas detalles de lo que se hizo.
\end{abstract}

\tableofcontents

\chapter{Introducción}

Introducción de la introducción.

Re-usar lo de la especialización.

\section{Objetivos}

Re-usar lo del pre-proyecto.

\section{Contribuciones}

Latencia.

GPU.

\section{Estructura}

Como siguen los capítulos y las secciones.

\chapter{Estado del Arte}

Introducción del capítulo.

\section{Análisis del Rendimiento}

\section{Herramientas de Soporte}

\section{Automatización}

\chapter{Descripción del Problema}

Introducción del capitulo.

\chapter{Propuesta de Solución}

Introducción del capitulo.

\section{Casos de Estudio}

\chapter{Conclusiones y Trabajo Futuro}

Introducción del capitulo.

\section{Conclusiones}

\section{Trabajo Futuro}

\bibliography{report}

\appendix

\chapter{Reporte de Ejemplo}

\end{document}

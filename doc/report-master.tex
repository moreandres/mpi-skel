\documentclass[a4paper]{report}

\usepackage{graphicx}
\usepackage{float}
\usepackage{hyperref}
\usepackage{xcolor}
\usepackage[spanish]{babel}
\usepackage{listings}
\usepackage{enumitem}
\usepackage[utf8]{inputenc}

\lstset{language=c, frame=tlrb, basicstyle=\scriptsize, breaklines=true, numberbychapter=false,numbers=left}
\setlist[enumerate]{noitemsep}
\setlist[itemize]{noitemsep}

\bibliographystyle{unsrt}

\begin{document}

\setlength{\parindent}{0cm}
\renewcommand{\tablename}{Tabla}

\title{Universidad Nacional de La Plata\\Facultad de Informática\\ \bigskip
{\large Tesis presentada para obtener el grado de \\}  Magister en Cómputo de Altas Prestaciones\\ \bigskip
  Infraestructura para el Análisis de Rendimiento}

\author{
  Alumno: Andrés More - {\tt amore@hal.famaf.unc.edu.ar}\\
  Director: Dr Fernando G. Tinetti - {\tt fernando@lidi.info.unlp.edu.ar}
}

\date{Octubre de 2014}

\maketitle

\begin{abstract}
Este trabajo revisa la construcción de una infraestructura de análisis de rendimiento para aplicaciones de cómputo de altas prestaciones; incluyendo casos de estudio con aplicaciones del mundo real.

\bigskip

Este trabajo contribuye con un generador de reporte automático de rendimiento para aplicaciones utilizando tecnología {\it OpenMP} sobre sistemas {\it GNU/Linux}.
\end{abstract}

\tableofcontents

\chapter{Introducción}

Este capítulo introduce el tema bajo estudio, definiendo los objetivos principales, las contribuciones logradas durante la investigación, y detallando la estructura del resto del documento.

\section{Objetivos}

La propuesta principal consiste en el desarrollo de una infraestructura de soporte para el análisis de aplicaciones de cómputo de altas prestaciones ({\it HPC}, por sus siglas en inglés). Este trabajo se realiza como extensión al trabajo final {\it Herramientas para el Soporte de Análisis de Rendimiento} de la Especialización en Cómputo de Altas Prestaciones y Tecnología Grid.

\bigskip

La infraestructura desarrollada implementa un procedimiento de análisis de rendimiento ejecutando pruebas de referencia, herramientas de perfil de rendimiento y análisis de resultados. La infraestructura genera como etapa final un informe detallado que soporta la tarea de optimización con información cuantitativa.
El reporte final incluye datos estadísticos de la aplicación y el sistema donde se ejecuta, además de gráficos de desviación de resultados, escalamiento de problema y cómputo e identificación de cuellos de botella.

\section{Contribuciones}

La siguiente lista enumera las diferentes publicaciones realizadas durante el cursado del magister y el desarrollo de la tesis.

\begin{enumerate}
\item Estudio de Multiplicación de Matrices. Reporte Técnico. Realizado como parte del curso {\it Fundamentos de Procesamiento Paralelo} dictado por {\it Marcelo Laiouf}.
\item Mejora de Latencia en Redes {\it Ethernet}. JAAIO. More, Garabato, Rosales.
\item Comparación de Implementaciones de una Operación BLAS. Reporte Técnico. Realizado como parte del curso {\it Programación GPU de Propósito General} dictado por {\it Margarita Amor}.
\item Sección {\it Intel Cluster Ready} y {\it Intel Cluster Checker} en el libro {\it Programming Intel Xeon Phi}. Intel Press. 2013.
\end{enumerate}

\section{Metodología}

En base al problema y a los objetivos establecidos previamente la metodología adoptada es la siguiente:

\section{Estructura}

El resto del documento se estructura de la siguiente forma:

\begin{itemize}
\item Capitulo 2: revisa el estado del arte de los temas incluidos.
\item Capitulo 3: describe el problema a resolver.
\item Capitulo 4: muestra la propuesta de solución.
\item Capitulo 5: aplica la solución a casos de estudio.
\item Capítulo 6: concluye reflejando los objetivos y proponiendo trabajo futuro.
\end{itemize}

\chapter{Estado del Arte}

Este capítulo revisa el estado del arte de los temas revisados en este tesis.

\section{Análisis del Rendimiento}

El análisis del rendimiento es un tema diverso, existen diferentes puntos de vista, ya sean diferentes niveles, componente, arquitecturas.

niveles,

componentes.

arquitecturas.

\section{Herramientas de Soporte}

Benchmarks.

Profilers.

Vectorizadores. Compiladores.

\section{Automatización}

Métodos generales.

\chapter{Descripción del Problema}

Este capítulo introduce el problema a resolver.

En el área de HPC los programadores son los mismos especialistas del dominio del problema a resolver. Las rutinas
más demandantes de cálculo son en su mayoría científicas y su alta complejidad hace posible su correcta implementación sólo por los mismos investigadores. Estas cuestiones resultan en un tiempo reducido de optimización de rendimiento
e impactan directamente en la productividad de los grupos de investigación y desarrollo. Frecuentemente el proceso de optimización termina siendo hecho de modo {\it ad-hoc}, sin la utilización de información cuantitativa para dirigir los
esfuerzos de mejora.

\chapter{Propuesta de Solución}

Este capítulo muestra la propuesta de solución, incluyendo el diseño de la misma a diferentes niveles, un ejemplo de su reporte de salida y realizando casos de aplicación con aplicaciones del mundo real.

\section{Infraestructura}

{\tt bottleneck}

\section{Diseño de Alto Nivel}

componentes.

\section{Diseño de Bajo Nivel}

comportamiento de cada componente.

\section{Reporte Generado}

secciones dentro del reporte, que datos incluyen y para qué son útiles.

\chapter{Casos de Aplicación}

\section{NEC}

Mario Trangoni.

\section{IUA}

Elementos Finitos.

\section{Espresso}

Biología.

\chapter{Conclusiones y Trabajo Futuro}

Este capítulo concluye revisando los objetivos propuestos y posibles líneas de investigación como continuación.

\section{Conclusiones}

\section{Trabajo Futuro}

Doctorado.

\bibliography{report}

\appendix

\chapter{Reporte de Ejemplo}

\end{document}

\documentclass[a4paper,twocolumn]{article}

\usepackage{graphicx}
\usepackage{float}
\usepackage{hyperref}
\usepackage[spanish]{babel}

\begin{document}

\title{Universidad Nacional de La Plata - Facultad de Inform\'atica\\
Especializaci\'on en C\'omputo de Altas Prestaciones\\
\bigskip
Propuesta de Trabajo Final\\
Herramientas para el Soporte de An\'alisis de Rendimiento}

\author{Alumno: Andr\'es More - {\tt amore@hal.famaf.unc.edu.ar}\\
Director: Fernando Tinetti - {\tt fernando@lidi.info.unlp.edu.ar}}

\date{Agosto 2010}

\twocolumn[
  \begin{@twocolumnfalse}
    \maketitle
    \begin{abstract}
      Este documento describe una propuesta de Trabajo Final para la
      Especializaci\'on en C\'omputo de Altas Prestaciones, dictada en la Facultad
      de Inform\'atica de la Universidad Nacional de La Plata. La propuesta
      principal consiste en el estudio de los m\'etodos existentes para
      evaluaci\'on y optimizaci\'on del rendimiento de aplicaciones de c\'omputo de
      altas prestaciones.

      \smallskip

      Se establecen m\'ultiples objetivos espec\'ificos para
      este trabajo, incluyendo la documentaci\'on de requerimientos en este tipo de
      aplicaciones, la generaci\'on de una taxonom\'ia de las tecnolog\'ias
      disponibles para an\'alisis, su posterior estudio y ejercitaci\'on, y 
      finalmente el desarrollo de un m\'etodo generalizado para an\'alisis de
      rendimiento, entre otros.

      \smallskip

      Luego de cumplir estos objetivos iniciales se plantea como cierre analizar
      varios n\'ucleos espec\'ificos de c\'alculo tom\'andolos como casos
      integradores . De este modo se podr\'a demostrar el procedimiento
      de an\'alisis de rendimiento utilizando las herramientas y metodolog\'ias
      estudiadas previamente.

      \smallskip

    \end{abstract}
  \end{@twocolumnfalse}
]

% \tableofcontents

\section{Motivaci\'on}

El desarrollo de aplicaciones que utilizan c\'omputo de alto rendimiento es
usualmente realizado por especialistas en el dominio especifico de los problemas
a simular o resolver, frecuentemente fuera del \'area de la inform\'atica.

\smallskip

Estos cient\'ificos no poseen todos los conocimientos, experiencia o recursos
necesarios para una implementaci\'on adecuada, implicando una reducci\'on del
tiempo disponible para el an\'alisis de los resultados obtenidos y su publicaci\'on.

\smallskip

Al no existir un m\'etodo general para el an\'alisis de rendimiento los
desarrolladores suelen utilizar m\'etodos de optimizaci\'on artesanales, raramente
teniendo las bases cuantitativas suficientes a la hora de realizar optimizaciones.

\section{Estado del Arte}

Actualmente existen diversas tecnolog\'ias para el an\'alisis de rendimiento,
ofreciendo diferentes niveles de portabilidad, aplicaci\'on o granularidad de la
informaci\'on obtenida.

\smallskip

La idea m\'as simple es utilizar el tiempo total de ejecuci\'on como medida de cuan
r\'apido se realiza una tarea, aunque no se obtiene informaci\'on sobre el lugar o
las consecuencias de potenciales optimizaciones.

\smallskip

Otra posibilidad es implementar en la aplicacion misma un sistema interno de
administracion de tiempo y recursos; en el caso de estar disponibles, tambi\'en se
pueden invocar a los mecanismos de an\'alisis de rendimiento en las librer\'ias de
terceros utilizadas dentro de la aplicaci\'on.

\smallskip

Una posibilidad mas interesante es utilizar un perfil de ejecucion extraido
dinamicamente para entender que partes del programa son utilizadas y cuales son
las que demandan mayor tiempo de ejecucion.

\smallskip

En el otro extremo, el hardware en que consiste una unidad de computo puede exportar
registros contadores de eventos que pueden ser accedidos para comprobar la
eficiencia de la ejecucion de un programa.

\section{Objetivos}

Ademas del estudio general del analisis de rendimiento, este trabajo tiene como
objetivos especificos:

\begin{enumerate}
\item Generar una taxonom\'ia de m\'etodos de analisis de rendimiento
\item Realizar una comparaci\'on de los mismos
\item Ejercitar estas herramientas, documentando pasos y problemas comunes
\item Proponer un proceso de analisis para aplicaciones de alto rendimiento
\item Documentar requerimientos generales de aplicacioness de altas prestaciones
\end{enumerate}

\section{Temas a Investigar}

Conceptos de An\'alisis de Rendimiento.
Modelo de Pipeline.
Herramientas.
oprofile \cite{oprofile}
gprof \cite{gprof}
mpi profiling interface, traces. \cite{mpi}
Comunicaci\'on: Balance, Comunicaci\'on.

\section{Plan de Actividades}

El siguiente cronograma muestra el plan de actividades tentativo.

\begin{figure}[H]
  \begin{center}
    \begin{tabular}{|l|l|}\hline
      {\bf Actividad} & {\bf Duraci\'on} \\ \hline
      Recolecci\'on de Material & Agosto \\ \hline
      Tecnolog\'ias & Setiembre/Octubre \\ \hline
      Procedimiento General & Noviembre \\ \hline
      An\'alisis de Ejemplos & Diciembre/Enero \\ \hline
      Resultados/Conclusiones & Febrero \\ \hline
    \end{tabular}
    \caption{Detalle de Actividades}
  \end{center}
  \label{schedule}
\end{figure}

\section{Trabajo Futuro}

Esta propuesta de trabajo es presentada como base de un desarrollo mayor a ser
realizado para el grado de Magister consistente en la construcci\'on conjunta de
una librer\'ia m\'as una infrastructura automatizada de an\'alisis de rendimiento.

\smallskip

Una vez entendidas las posibilidades tecnol\'ogicas y estudiado su aplicaci\'on en
multiples casos de ejemplo se puede entonces emprender una generalizacion de los
requerimientos de este tipo de aplicaciones y del proceso mismo de analsis.

\section{Material de Referencia}

Existe abundante material sobre el tema, desde textos base, tecnolog\'ias
espec\'ificas o herramientas en particular.

\appendix

\section{Ejemplos Integradores}

Utilizando las t\'ecnicas estudiadas, se propone implementar y luego analizar las
siguientes aplicaciones:

\begin{itemize}

\item \emph{matrix}: multiplicaci\'on de dos matrices densas, utilizando operaciones
  de punto flotante en forma secuencial.

\item \emph{heat2d}: distribuci\'on de calor en dos dimensiones, utilizando
  iteraciones de operaciones simples que simulan el paso del tiempo.

\item \emph{fft}: aplicaci\'on de transformadas de {\it Fourier}, utilizando
  operaciones complejas de punto flotante.

\item \emph{queens}: colocaci\'on de N reinas en un tablero de
  ajedrez sin amenaza mutua, utilizando operaciones enteras sobre arreglos.

\end{itemize}

\begin{thebibliography}{9}
  
\bibitem{mpi}
  Message Passing Interface Forum,
  \emph{MPI: A Message-Passing Interface Standard},
  2.2,
  2009.

\bibitem{openmp}
  OpenMP Architecture Review Board,
  \emph{OpenMP Application Program Interface}.
  3.0,
  2008.

\bibitem{tinetti}
  Fernando Tinetti,
  \emph{Computo Paralelo en Redes Locales de Computadoras},
  2004.

\bibitem{gprof}
 Susan L. Graham,  Peter B. Kessler,  Marshall K. McKusick,
 \emph{gprof: A Call Graph Execution Profiler},
 1982.

\bibitem{oprofile}
J. Levon,
\emph{oprofile: hardware profiler for Linux systems},
{\tt http://oprofile.sourceforge.net}.

\bibitem{hennessy-patterson}
 John. L. Hennesy, David A. Patterson,
 \emph{Computer Architecture: A Quantitative Approach, 3rd Edition},
 2002.

\bibitem{intel}
 Intel Press,
 \emph{Intel64 and IA-32 Architectures Software Developer's Manual - Volume 3B: System Programming Guide, Part 2},
 March 2010.

\bibitem{what}
 Ulrich Deeper,
 \emph{What Every Programmer Should Know About Memory},
 Novermber 2007.

\bibitem{patterns}
 G. Mattson, B.A. Sanders and B.L. Massingill, 
 \emph{Patterns for Parallel Programming, Addison-Wesley},
 2004.

\bibitem{automatic-performance-analysis}
 T. Margalef, J. Jorba, O. Morajko, A. Morajko, E. Luque,
 \emph{Different approaches to automatic performance analysis of distributed applications},
 2004.

\bibitem{capturing-performance-knowledge}
 K. Huck, O. Hernandez, V. Bui, S. Chandrasekaran, B. Chapman, A. Malony, L McInnes, B. Norris,
 \emph{Capturing performance knowledge for automated analysis},
 2008.

\bibitem{automatic-openmp-mpi-analysis}
 F. Wolf, B. Mohr,
 \emph{Automatic performance analysis of hybrid MPI/OpenMP applications},
 2003.

\bibitem{intro-software-performance}
 C. Smith,
 \emph{Introduction to software performance engineering: origins and outstanding problems},
 2007.

\bibitem{future-software-performance}
 M. Woodside, G. Franks, D. Petriu,
 \emph{The Future of Software Performance Engineering},
 2007.

\bibitem{critical-overview}
 J. Browne,
 \emph{A critical overview of computer performance evaluation},
 1976.

\bibitem{hpctoolkit}
  Rice University,
 \emph{HPC Toolkit},
 {\tt http://hpctoolkit.org}.

\bibitem{papi}
  TBD,
  \emph{Performance Application Programming Interface},
  {\tt http://icl.cs.utk.edu/papi}.

% http://en.wikipedia.org/wiki/Performance_tuning

\end{thebibliography}

\end{document}

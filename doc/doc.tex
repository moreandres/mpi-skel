\documentclass[a4paper,twocolumn]{article}

\usepackage{graphicx}
\usepackage{float}
\usepackage{hyperref}
\usepackage[spanish]{babel}

\begin{document}

\title{Universidad Nacional de La Plata - Facultad de Inform\'atica\\Especializaci\'on en C\'omputo de Altas Prestaciones\\ \bigskip Propuesta de Trabajo Final\\Herramientas para el Soporte de Analisis de Rendimiento}
\author{ Alumno: Andr\'es More - {\tt amore@hal.famaf.unc.edu.ar}\\Director: Fernando Tinetti - {\tt fernando@lidi.info.unlp.edu.ar}}
\date{Julio 2009}

\twocolumn[
  \begin{@twocolumnfalse}
    \maketitle
    \begin{abstract}
      Este documento describe una propuesta de trabajo final para la especializaci\'on en c\'omputo de altas prestaciones dictada en la facultad nacional de la plata.
      La propuesta consiste en el estudio de los metodos existentes para evaluacion y optimizacion del rendimiento de programas.
      A manera de ejemplificacion, se analizan diversos casos integradores de aplicaciones representativas, representando un nucleo especifico de calculo. De este modo se repite un
      procedimiento de analisis de rendimiento utilizando diferentes herramientas y metodologias.
    \end{abstract}
  \end{@twocolumnfalse}
]

\tableofcontents

\section{Objetivos}

Generar una taxonomia de m\'etodos de analisis de rendimiento.
Realizar una comparaci\'on de los mismos.
Ejercitar estas herramientas, documentando pasos y problemas comunes.
Proponer un proceso de analisis para aplicaciones de alto rendimiento.
Generar infrastructura que soporte automaticamente el analisis de performance.
Generar una librer\'ia que facilite el desarollo de aplicaciones de altas prestaciones.
Documentar requerimientos generales de aplicacioness de altas prestaciones.
Generalizar las aplicaciones utilizando un pipeline.

\section{Motivacion y Estado del Arte}

MPI.
Expertos en Dominio.
Analisis de Rendimiento es artesanal.

\section{Temas a Investigar}

Analisis de Rendimiento.
Herramientas.

\section{Material Bibliografico}

La gran mayor\'ia de las referencias son a su vez material a utilizar durante la investigaci\'on.

\begin{enumerate}
\item{Linux Device Drivers}. Detalles de implementaci\'on de subsistemas utilizando patrones de reuso.
\item{Design Patterns}. Base de todo estudio de patrones de dise\~no
\item{G. Mattson, B.A. Sanders and B.L. Massingill, Patterns for Parallel Programming, Addison-Wesley, 2004}.
\end{enumerate}

\section{Plan de Actividades}

\appendix

\section{Ejemplos Integradores}

\subsection{matrix}

La aplicacion matrix realizara una multiplicacion de dos matrices densas.
operaciones de punto flotante en forma secuencial.

\subsection{queens}

La aplicacion queens realizara la colocacion de 8 reinas en un tablero de ajedrez sin que las mismas se amenacen mutuamente.
arreglos de enteros.

\subsection{heat2d}

Distribuci\'on de calor en dos dimensiones.
operaciones simples en iteraciones.

\subsection{fft}

Utilizacion de transformadas de fourier.
operaciones altamente complejas de punto flotante.

\begin{thebibliography}{9}
  
\bibitem{mpi}
  Message Passing Interface Forum,
  \emph{MPI: A Message-Passing Interface Standard},
  2.2,
  2009.

\bibitem{openmp}
  OpenMP Architecture Review Board,
  \emph{OpenMP Application Program Interface}.
  3.0,
  2008.

\bibitem{tinetti}
  Fernando Tinetti,
  \emph{Computo Paralelo en Redes Locales de Computadoras},
  2004.

\bibitem{gprof}
Susan L. Graham,  Peter B. Kessler,  Marshall K. McKusick,
\emph{gprof: A Call Graph Execution Profiler},
1982.

\bibitem{hennessy-patterson}
John. L. Hennesy, David A. Patterson,
\emph{Computer Architecture: A Quantitative Approach, 3rd Edition},
2002.

\bibitem{intel}
Intel Press,
\emph{Intel64 and IA-32 Architectures Software Developer's Manual - Volume 3B: System Programming Guide, Part 2},
March 2010.

\bibitem{what}
Ulrich Deeper,
What Every Programmer Should Know Abot Memory,
Novermber 2007.

\end{thebibliography}

\end{document}
